\documentclass[12pt,a4paper]{article}

% Essential packages
\usepackage[utf8]{inputenc}
\usepackage[T1]{fontenc}
\usepackage{graphicx}
\usepackage{amsmath,amssymb}
\usepackage{booktabs}
\usepackage{hyperref}
\usepackage{geometry}

% Page geometry
\geometry{margin=2.5cm}

% Document information
\title{Robot Arm Project Report}
\author{Your Name}
\date{\today}

\begin{document}

% Title page
\begin{titlepage}
    \centering
    \vspace*{1cm}
    {\Huge\textbf{Robot Arm Project Report}\par}
    \vspace{2cm}
    {\Large\textit{Project for the introduction to robotics course}\par}
    \vspace{3cm}
    {\Large Luca Sartore\par}
    \vfill
    University of Trento
    \vspace{1cm}
\end{titlepage}

% Table of contents
\tableofcontents
\newpage

% Abstract
\section*{Abstract}
\addcontentsline{toc}{section}{Abstract}
In this report I will describe the project ``robot arm''.
this project consisted in the creation fo a giraffe robot that 
the task of placing the microphone in front of a speaker inside a conference room.

All the source code, as well as the original assignment can be found in the 
github repository \url{https://github.com/lucaSartore/RobotArm}

\section{Introduction}
\subsection{Background}
\subsection{Objectives}
\subsection{Scope}

% % Example of a figure
% \begin{figure}[htbp]
%     \centering
%     \includegraphics[width=0.7\textwidth]{images/f1.png}
%     \caption{Robot Arm Design}
%     \label{fig:robotarm}
% \end{figure}


\end{document}